\documentclass{article}
\usepackage{ngerman}
\usepackage{graphicx}
\usepackage{xfrac}
\usepackage{float}
\usepackage{tabularx}
\usepackage[margin=1in]{geometry}
\usepackage{fancyhdr}
\usepackage{lmodern}

\newcommand{\titel}{Leipzig eSports e.V. Corporate Design Manual}
\newcommand{\autor}{Leipzig eSports e.V.}
\newcommand{\version}{0.2}
\usepackage{xcolor}
\definecolor{LESOrange}{HTML}{FF7929}
\definecolor{LESBlue}{HTML}{2E72C9}
\definecolor{LESYellow}{HTML}{FFD129}

\definecolor{SECLight}{HTML}{b4b8bd}
\definecolor{SECMedium}{HTML}{666666}
\definecolor{SECDark}{HTML}{333333}

\newcommand\cfield[1][black]{\textcolor{#1}{\linebreak\noindent\rule{\textwidth}{1cm}\linebreak}}
\usepackage{ifxetex,fontspec}
\usepackage{titlesec}

\setsansfont{Kanit}
\setmainfont{Titillium Web}

\newfontfamily\headingfont{Kanit}

\titleformat*{\section}{\Large\headingfont}

\setlength{\parindent}{0pt}

\newcommand\alphabet{ABCDEFGHIJKLMNOPQRSTUVWXYZ\\abcdefghijklmnopqrstuvwxyz}

\newcommand\fonttest[1]{\begin{samepage}{\fontsize{18pt}{18pt}\fontspec{#1}\textbf{\alphabet}\\\alphabet}\end{samepage}}


\pagestyle{fancy}
\begin{document}

\begin{titlepage}
\vspace*{\fill}
\title{\titel}
\author{\autor}
\includegraphics[width=\textwidth]{Docs/Logo.eps}
{\let\newpage\relax\maketitle}
\begin{center}
Version \version
\end{center}
\vspace*{\fill}
\end{titlepage}

\cleardoublepage

\tableofcontents

\cleardoublepage
\section{Einführung}

Das folgende Dokument beschreibt das Corporate Design des Vereins Leipzig eSports e.V. und dient als Grundlage für die Gestaltung aller Dokumente die den Verein repräsentieren.

\section{Der Name und das Kürzel}

Der Verein tritt in der Regel mit dem vollen Vereinsnamen \emph{Leipzig eSports e.V.} auf. Zur Verbesserung des Leseflusses kann das \emph{e.V.} in nicht-formellen Texten wie Artikeln, Blog-Einträgen oder Forenposts weggelassen werden.

Das Kürzel für Clan- oder Gruppenstrukturen in Spielen, Netzwerken, u.ä. lautet immer \emph{LES}. Aus technischen Gründen (Mindestlänge, bereits vergeben) kann es hier notwendig sein, dass von diesem Kürzel abgewichen wird. Ansonsten sind andere Schreibweisen oder Abkürzungen nicht zulässig.

\section{Das Logo}

Das Logo zeigt den vollen Schriftzug \emph{Leipzig eSports} in Gelb und Blau und den stilisierten Löwen in einer Wappenform.
\begin{figure}[H]
\includegraphics[width=\textwidth]{Docs/Logo.eps}
\caption{Logo mit Schriftzug}
\end{figure}

Die Schriftart ist Helvetica Neue bei der das L abgewandelt ist um der Silhouette des City-Hochhauses zu ähneln.

\begin{figure}[H]
\centering
\includegraphics[width=0.33\textwidth]{Docs/logo_detail_L.png}
\caption{Detailansicht L}
\end{figure}


Als weitere Variante gibt es das Logo ohne Schriftzug, nur als Wappen. Es bietet sich an für Icons und Dokumente in denen nicht viel Platz ist.

\begin{figure}[H]
\centering
\includegraphics[width=0.33\textwidth]{Docs/Emblem.eps}
\caption{Logo ohne Schriftzug}
\end{figure}

\subsection{Platzierung}

Um das Logo herum ist ein Abstand einzuhalten der mindestens $\sfrac{1}{7}$ der Höhe des Logos entspricht. Aufgrund der transparenten Bereiche ist das Platzieren auf Hintergründen mit mehreren Farben oder starken Kontrasten nicht sinnvoll. Das Logo sollte idealerweise auf sehr hellen oder sehr dunklen, einfarbigen Hintergründen platziert werden, um optimalen Kontrast zu garantieren.

\begin{figure}[H]
\begin{centering}
\hfill
\fbox{\includegraphics[width=0.3\textwidth]{Docs/logo_placement_bright.png}}
\fbox{\includegraphics[width=0.3\textwidth]{Docs/logo_placement_medium.png}}
\fbox{\includegraphics[width=0.3\textwidth]{Docs/logo_placement_dark.png}}\hfill

\hfill
\fbox{\includegraphics[width=0.3\textwidth]{Docs/logo_placement_picture.png}}
\fbox{\includegraphics[width=0.3\textwidth]{Docs/logo_placement_blur.png}}\hfill
\end{centering}
\caption{Beispiele für Kontrast auf verschiedenen Hintergründen}
\end{figure}

\section{Die Schriften}

\begin{samepage}
\subsection{Titillium Web}
\fonttest{Titillium Web}

Titillium Web ist die Standardschrift die für alle Texte genutzt wird.
\end{samepage}

\subsection{Kanit}
\begin{samepage}
\fonttest{Kanit}

Kanit wird für alle Überschriften und Titel und ausschlieslich in Fett verwendet.
\end{samepage}

\section{Die Farben}

\subsection{Primärfarben}
\begin{samepage}
\cfield[LESOrange]
Orange
\begin{itemize}
\item HTML \#FF7929
\item RGB 255/121/41
\item HSV 22/84/100
\item CMYK 0/53/84/0
\end{itemize}
\end{samepage}

\begin{samepage}
\cfield[LESBlue]
Blau
\begin{itemize}
\item HTML \#2E72C9
\item RGB 46/115/201
\item HSV 214/77/79
\item CMYK 77/43/0/21
\end{itemize}
\end{samepage}

\begin{samepage}
\cfield[LESYellow]
Gelb
\begin{itemize}
\item HTML \#FFD129
\item RGB 255/209/41
\item HSV 47/84/100
\item CMYK 0/18/84/0
\end{itemize}
\end{samepage}

Die Primärfarben orientieren sich an den bisher benutzten Vereinsfarben und sind nur leicht abgeändert, um etwas kräftiger zu wirken. Sie finden sich nicht nur im Logo sondern auch in hervorhebenden Gestaltungselementen.

\subsection{Sekundärfarben}
\begin{samepage}
\cfield[SECLight]
Helles Grau
\begin{itemize}
\item HTML \#b4b8bd
\item RGB 180/184/189
\item HSV 213/5/74
\item CMYK 3/1/0/27
\end{itemize}
\end{samepage}

\begin{samepage}
\cfield[SECMedium]
Mittleres Grau
\begin{itemize}
\item HTML \#555555
\item RGB 85/85/85
\item HSV 0/0/33
\item CMYK 0/0/0/67
\end{itemize}
\end{samepage}

\begin{samepage}
\cfield[SECDark]
Dunkles Grau
\begin{itemize}
\item HTML \#333333
\item RGB 51/51/51
\item HSV 0/0/20
\item CMYK 0/0/0/80
\end{itemize}
\end{samepage}

Die Sekundärfarben dienen hauptsächlich als Hintergrund- oder Füllfarben. Das helle Grau hat eine leichte Blaukomponente und ist eine hellere, entsättigte Version des Vereinsblaus. Das dunkle Grau bildet einen guten Kontrast du Logo und Gestaltungselementen und sollte für diese als Hintergrund verwendet werden. Das mittlere Grau ist sowohl als Hintergrund von Gestaltungselementen, als auch als Übergang oder Rand für Elemente in den anderen Grautönen zu benutzen.

\section{Gestaltungselemente}

\subsection{Der Winkel}
Alle Elemente deren Linien nicht parallel zu den Dokumenträndern verlaufen, sollten einen Winkel von etwa 35° haben. Dieser Winkel entspricht dem der Diagonalen durch das Auge des Löwens im Wappen.\footnote{Der Winkel im Auge entspricht nicht exakt 35°, sondern weist einige Dezimalstellen auf. Der Einfachheit halber wird hier jedoch gerundet.}

%illustrationen/grafiken hier

\subsection{Hintergründe}
Hintergründe sollten so wenig Kontrast wie möglich aufweisen, um nicht mit Inhalten und Gestaltungselementen zu kollidieren. Es bieten sich flache Hintergründe in den Sekundärfarben und weichgezeichnete oder entsättigte Fotos an. Für besondere Veranstaltungen oder thematisch speziell einzuordnende Erzeugnisse, beispielsweise spielspezifische Streamoverlays, können jedoch auch andere, kontrastarme Hintergründe verwendet werden.

% beispielgrafiken hier

\subsection{Textbox}
Die Textbox dient vor allem der Abgrenzung von Text und Hintergrund. Sie besteht aus einem rechteckigen Feld in mittlerem Grau, das von zwei Streifen oben und unten begrenzt wird. Die Streifen sind rechts oben und links unten angewinkelt. Der obere Streifen hat immer eine der Primärfarben, der untere kann dunkles Grau oder auch eine Primärfarbe haben, je nach Hintergrund. Die Höhe der Streifen varriert nach Ausgabemedium und Auflösung, sollte aber $\sfrac{1}{4}$ der Gesamthöhe der Textbox nicht übersteigen. Einige Beispielwerte werden in der Tabelle gegeben.

\begin{table}[hb]
\begin{center}
\begin{tabular}{ | l | r | }
\hline
  Anwendung & Höhe \\
\hline
  Website & 8px \\
\hline
  Social Media Bilder & 8px \\
\hline
\end{tabular}
\end{center}
\caption{Beispielmaße für verschiendene Anwendungsfälle der Textbox}
\end{table}

%mehr beispiele im Laufe der Zeit ansammeln, nachdem sich herausgestellt hat, was gut aussieht

\end{document}